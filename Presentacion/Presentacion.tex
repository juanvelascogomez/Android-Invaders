\documentclass{beamer}

%\usetheme{Goettingen}
%\usetheme{Berkeley}
%\usetheme{Boadilla}
\usetheme{Warsaw}

\usepackage[utf8]{inputenc}
\usepackage[spanish]{babel}

%en corchetes ponemos nombres más cortos, para mejor visualización de las cabeceras del documento
\title[Charla Android]{Android Invaders}
\author[LPB]{$@asolisi, @juanvelascogomez y @$neon$_$520}
\institute[UGR]{Universidad de Granada}

\begin{document}

\maketitle


%los comandos con asterisco evitan la numeración
\section*{Introducción}

\begin{frame}
\begin{itemize}[<+-|alert@+>]
\item Este taller está pensado como pequeña introducción al \LaTeX 

\item Intentaremos dar algunas pequeñas pinceladas sobe su uso 

\item Para m\'as detalles véase la documentación de beamer.
\end{itemize}
\end{frame}

\begin{frame}
\begin{itemize}
\item<1-> Primero
\item<2-> Segundo
\item<3,4>Tercero
\item<2> Otro
\item<5-> Final
\end{itemize}
\end{frame}

\section{Otra sección}

\begin{frame}{Otra cosa mariposa}
\begin{block}{Bloques}
En beamer existen varios tipos de bloques....
Y en ellos podemos escribir lo que queramos.
\end{block}

Podemos poner texto fuera de los bloques

\begin{alertblock}{Bloques de alerta}
En éstos escribimos cosas a resaltar.
\end{alertblock}

\begin{exampleblock}{Ejemplos}
Y en estos los ejemplos, por ejemplo.
\end{exampleblock}

\end{frame}

\section{Pausas}

\begin{frame}

Hagamos lo mismo, con pausas.

\begin{block}{Bloques}
En beamer existen varios tipos de bloques....
Y en ellos podemos escribir lo que queramos.
\end{block}

\pause 
Podemos poner texto fuera de los bloques

\pause
\begin{alertblock}{Bloques de alerta}
En éstos escribimos cosas a resaltar.
\end{alertblock}

\pause
\begin{exampleblock}{Ejemplos}
Y en estos los ejemplos, por ejemplo.
\end{exampleblock}

\end{frame}

\section{Sobre}

\subsection{Impresiones}

\begin{frame}

Más cosas flotando...

\begin{exampleblock}{Ejemplos}
\begin{overprint}

\onslide<1>

Y en estos los ejemplos, por ejemplo.

\onslide<2>

Y en estos los ejemplos, por ejemplo. Ligeramente modificado.

\onslide<3>

O cualquier otra cosa.

\end{overprint}
\end{exampleblock}

\end{frame}


%el número 10 no es significativo, realmente dice que el número de citas es un número de a lo sumo dos dígitos
\begin{thebibliography}{10}
\bibitem{lshort} Tobias Oetiker, Hubert Partl, Irene Hyna and Elisabeth Schlegl, The not so short introduction to \LaTeX2e, \href{http://www.ctan.org/tex-archive/info/lshort/english/lshort.pdf}{ctan.org}.
\end{thebibliography}

\end{document}