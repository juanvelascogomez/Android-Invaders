\documentclass{beamer}
\usepackage{graphicx}
\graphicspath{ {images/} }
\usepackage{float}
\usepackage{wrapfig}

%\usetheme{Goettingen}
%\usetheme{Berkeley}
%\usetheme{Boadilla}
\usetheme{Warsaw}

\usepackage[utf8]{inputenc}
\usepackage[spanish]{babel}

%en corchetes ponemos nombres más cortos, para mejor visualización de las cabeceras del documento
\title[Charla Android]{Android Invaders}
\author[LPB]{$@asolisi, @juanvelascogomez y @neon520$}
\institute[UGR]{Universidad de Granada}

\begin{document}

\maketitle

%los comandos con asterisco evitan la numeración
\section*{Introducción}

\begin{figure}[h]
\includegraphics[width=11cm, height=8cm]{androidInvaders.png}
\end{figure}

%Primer grupo: Introduccion al problema
\begin{frame}{¿Dónde está el problema?}
	\begin{itemize}[<+-|alert@+>]
		
	\item Los dispositivos y entornos móviles son un objetivo y están constantemente amenazados por:
	\item Comportamientos del usuario arriesgados (y habituales)
	\item Vulnerabilidades de seguridad
	\item Dar a las actualizaciones la importancia que se merecen
	
	\end{itemize}
\end{frame}

%Segundo grupo: Actualizaciones del sistema
\begin{frame}{Actualizaciones del sistema}
	Algunas preguntas a resolver...
	
\begin{itemize}[<+-|alert@+>]
	
	\item ¿Disponéis de la última versión del sistema operativo para vuestro móvil?
	\item ¿Está disponible esa nueva versión para nuestro dispositivo?
	\item ¿Resuelve esto todas las vulnerabilidades conocidas?
	\item ¿Y las que no son públicas?
	
\end{itemize}
\end{frame}

%Tercer grupo: Vulnerabilidades Android
\begin{frame}{Versiones y vulnerabilidades en Android}
	
	\begin{block}{Android: 8 años}
		2008: 1.0
		
		2009: 1.1, 1.5, 1.6, 2.0
		
		2010: 2.1, 2.2, 2.3.x
		
		2011: 2.4.x, 3.x,4.0
		
		2012: 4.1, 4.2
		
		2013: 4.3, 4.4
		
		2014: 5.0
		
		2015: 5.1, 6.0
	\end{block}
		
\end{frame}

%Cuarto grupo: Vulnerabilidades Android II
\begin{frame}{Versiones y vulnerabilidades en Android}
	
	\begin{alertblock}{Número oficial de vulnerabilidades}
		
		Desconocido (hasta Agosto 2015 para móviles Nexus)
		
		\begin{itemize}[<+-|alert@+>]
			
			\item Agosto de 2015: 6
			\item Septiembre de 2015: 9
			\item Octubre de 2015: 30
			\item Noviembre de 2015: 7
			\item TOTAL: 52 vulnerabilidades conocidas
			
		\end{itemize}
	\end{alertblock}
	
\end{frame}

%Cuarto grupo: Usuarios y contraseñas
\begin{frame}{Sobre los usuarios y contraseñas}
	
	\begin{itemize}[<+-|alert@+>]
		
		\item \textbf{¿Tenéis cuenta de usuario en la plataforma del fabricante del dispositivo móvil?}
		\item \textbf{¿Vuestro usuario es conocido?}
		\item ¿Seguro?
		\item ¿La contraseña es robusta?
		\item ¿La reutilizáis
		
	\begin{block}{TOP de contraseñas más usadas en 2015}
		123456
		
		password
		
		12345678
		
		qwerty
		
		iloveyou
		
	\end{block}
	
	\end{itemize}
	
\end{frame}


\section{Camuflar un APK}

\begin{frame}{Otra cosa mariposa}
\begin{block}{Bloques}
En beamer existen varios tipos de bloques....
Y en ellos podemos escribir lo que queramos.
\end{block}

Podemos poner texto fuera de los bloques

\begin{alertblock}{Bloques de alerta}
En éstos escribimos cosas a resaltar.
\end{alertblock}

\begin{exampleblock}{Ejemplos}
Y en estos los ejemplos, por ejemplo.
\end{exampleblock}

\end{frame}

\section{Prueba de concepto real}

\begin{frame}

Hagamos lo mismo, con pausas.

\begin{block}{Bloques}
En beamer existen varios tipos de bloques....
Y en ellos podemos escribir lo que queramos.
\end{block}

\pause 
Podemos poner texto fuera de los bloques

\pause
\begin{alertblock}{Bloques de alerta}
En éstos escribimos cosas a resaltar.
\end{alertblock}

\pause
\begin{exampleblock}{Ejemplos}
Y en estos los ejemplos, por ejemplo.
\end{exampleblock}

\end{frame}

\section{¿Y si salimos afuera?}

\subsection{Impresiones}

\begin{frame}

Más cosas flotando...

\begin{exampleblock}{Ejemplos}
\begin{overprint}

\onslide<1>

Y en estos los ejemplos, por ejemplo.

\onslide<2>

Y en estos los ejemplos, por ejemplo. Ligeramente modificado.

\onslide<3>

O cualquier otra cosa.

\end{overprint}
\end{exampleblock}

\end{frame}

%el número 10 no es significativo, realmente dice que el número de citas es un número de a lo sumo dos dígitos
\begin{thebibliography}{10}
\bibitem{lshort} Tobias Oetiker, Hubert Partl, Irene Hyna and Elisabeth Schlegl, The not so short introduction to \LaTeX2e, \href{http://www.ctan.org/tex-archive/info/lshort/english/lshort.pdf}{ctan.org}.
\end{thebibliography}

\end{document}
